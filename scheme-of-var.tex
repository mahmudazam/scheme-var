
\subsection{The Scheme of a Variety}

For all results, we will consider an algebraically closed field
$k$. We will then follow definitions from \cite{HartAG}.

\begin{lem}\label{lem:affine-var-to-scheme}
Let $U$ be an affine $k$--variety; $I(U)$, its ideal, and
$A = A(U) = k[x_1, \dots, x_n]/I(U)$,
its coordinate ring. We then have a function:
\[
\beta_U : t(U) \to \Spec(A)
        : C \mapsto q(I(C))
\]
where $I(C) = \set[f \in {k[x_1, \dots, x_n]}]{\forall c \in C, f(c) = 0}$ is
the ideal of $C$ and $q : k[x_1, \dots, x_n] \to A$ is the quotient map.
For a closed subset $D \subset U$, we have:
\[
\beta_U(t(D)) = V(q(I(D))) = \set[P \in \Spec(A)]{q(I(D)) \subset P}
\]
Furthermore, we have a function:
\[
\gamma_U : \Spec(A) \to t(U) : P \mapsto Z(q^{-1}(P))
\]
where $Z(P) = \set[x \in k^n]{\forall f \in P, f(x) = 0}$.
For any ideal $K \subset A$, we have:
\[
\gamma_U(V(K)) = t(Z(q^{-1}(K)))
\]
Finally, $\beta_U$ and $\gamma_U$ are inverses to each other.
\end{lem}
\begin{proof}
We first check that the map $\beta_U$ is well defined.
Let $R = k[x_1, \dots, x_n]$.
Given $C \in t(U)$, $I(C) \supset I(U)$ since $I(U)$ is the set of all
$f \in R$ such that for all $x \in U$, $f(x) = 0$.
Then, $I(C)$ is prime by \cite[Cor. I.1.4]{HartAG} because $C$ is irreducible.
Then, $I(C)/I$ is prime in $A$ by \cref{thm:correspondence-cring}.

Let $C \in t(D)$. Then, $\beta_U(C) = q(I(C))$. Since $C \subset D$,
$I(D) \subset I(C)$. Furthermore, $I(C)$ is prime by \cite[Cor. I.1.4]{HartAG}.
Then, $q(I(C))$ is prime and $q(I(D)) \subset q(I(C))$ by
\cref{thm:correspondence-cring}. Thus, $\beta_U(C) = q(I(C)) \in V(q(I(D)))$.

Now, suppose $P \in V(q(I(D)))$. Then, $P = q(q^{-1}(P))$ but since
$q^{-1}(P)$ is prime and contains $q^{-1}(q(I(D))) = I(D)$,
$Z(q^{-1}(P))$ is an irreducible subset of $Z(I(D)) = D$
by \cite[Cor. I.1.4]{HartAG}, and, of course, $Z(q^{-1}(P))$ is closed
in $k^n$ and hence also in $D$ by the definition of Zariski topology.
Hence, then $\beta_U(Z(q^{-1}(P))) = q(I(Z(q^{-1}(P)))) = q(q^{-1}(P)) = P$.
Thus, we have shown that $P \in \beta_U(t(D))$. Therefore,
$\beta_U(t(D)) = V(q(I(D)))$.

Now, we show that $\gamma_U$ is well defined. Since $P$ is prime,
so is $q^{-1}(P)$ by \cref{thm:correspondence-cring}, and
$Z(q^{-1}(P))$ is an irreducible closed subset of $k^n$
by \cite[Cor. I.1.4]{HartAG}. However, $I(U) \subset q^{-1}(P)$
by \cref{thm:correspondence-cring} so that $Z(q^{-1}(P)) \subset Z(I(U)) = U$.
Then, by \cref{cor:irred-indep} and \cref{cor:op-op-cls-cls},
$Z(q^{-1}(P)) \in t(U)$.

Let $P \in V(K)$. Then, $K \subset P$ implies $q^{-1}(K) \subset q^{-1}(P)$
which, in turn, implies $Z(q^{-1}(P)) \subset Z(q^{-1}(K))$. However,
since $Z(q^{-1}(P))$, it is irreducible and closed in
$U$ which by two applications of \cref{cor:irred-indep} and
\cref{cor:op-op-cls-cls}, shows that $Z(q^{-1}(P))$ is
irreducible and closed in $Z(q^{-1}(K))$. Thus,
$\gamma_U(P) = Z(q^{-1}(P)) \in t(Z(q^{-1}(K)))$.

Now, let $C \in t(Z(q^{-1}(K)))$. Then, $C$ is an irreducible
closed subset of $Z(q^{-1}(K))$ which, by \cref{cor:irred-indep}
and \cref{cor:op-op-cls-cls}, implies that $C$ is an irreducible
closed subset of $k^n$ and hence, $I(C)$ is prime, by
\cite[Cor. I.1.4]{HartAG}. Next, $q^{-1}(K) \subset I(C)$,
because if $f \in k[x_1, \dots, x_n]$ vanishes on $Z(q^{-1}(K))$,
it vanishes on every subset thereof.
Then, by \cref{thm:correspondence-cring},
$K = q(q^{-1}(K)) \subset q(I(C))$ and $q(I(C))$ is prime in $A$, so that
$q(I(C)) \in V(K)$. Next, $\gamma_U(q(I(C))) = Z(q^{-1}(q(I(C)))) = C$.
Thus, $C \in \gamma_U(V(K))$.

We finally observe that, by \cite[Cor. I.1.4]{HartAG} and
\cref{thm:correspondence-cring},
\[
\beta_U(\gamma_U(P)) = q(I(Z(q^{-1}(P)))) = P
\]
and
\[
\gamma_U(\beta_U(C)) = Z(q^{-1}(q(I(C)))) = C
\]
\end{proof}

\begin{cor}\label{cor:schemat-affine-var-homeo}
The maps $\beta_U$ and $\gamma_U$ of \cref{lem:affine-var-to-scheme}
are continuous inverses of each other and hence, we have a homeomorphism:
\[
\beta_U : t(U) \stackrel{\cong}{\leftrightarrows} \Spec(A) : \gamma_U
\]
\end{cor}
\begin{proof}
By \cref{lem:affine-var-to-scheme}, it suffices to observe that
both maps take closed sets to closed sets.
\end{proof}

\begin{lem}\label{lem:affine-var-T1}
Every affine variety over $k$ is $T_1$.
\end{lem}
\begin{proof}
Let $U$ be an affine $k$--variety such that $U$ is an irreducible
Zariski closed subset of $k^n$. That is, $U = Z(I(U))$ for some
ideal $I(U) \subset k[x_1, \dots, x_n]$. Then, for any
$a = (a_1, \dots, a_n) \in U \subset k^n$,
$\set{a} = Z((x_1 - a_1, \dots, x_n - a_n))$ which is closed
in $k^n$.
Thus, $\set{a}$ is a closed subset of $U$ as well by \cref{cor:op-op-cls-cls}.
\end{proof}

\begin{lem}\label{lem:max-neighbour}
Let $R = k[x_1, \dots, x_n]$; $I \subset R$, an ideal, and $A = R/I$.
For every $P \in \Spec(A)$, and every
open neighbourhood $U$ of $P$ in $\Spec(A)$,
there is an $a \in k^n$ such that $M = q(\ker{\ev_a}) \subset A$ is a
maximal ideal, $M \in U$ and $P \subset M$.
\end{lem}
\begin{proof}
Since $U$ is open, there is some ideal $K \subset A$ such that
$U = \Spec(A) \setminus V(K)$. Then, $P \in U$ implies that
$P \not\supset K$. This means that there exists $f + I \in K$ such that
$f + I \not\in P$.
Let $q : R \to A$ be the quotient map.
If $f$ is identically zero on $Z(q^{-1}(P))$, then by
\cite[Cor. I.1.4]{HartAG},
\begin{align*}
        & Z(q^{-1}(P)) \subset Z(q^{-1}((f + I))) \\
\implies& q^{-1}(P) \supset I(Z(q^{-1}((f + I)))) \\
\implies& q^{-1}(P) \supset (q^{-1}(f + I)) \supset (f)
\end{align*}
so that $f \in q^{-1}(P) \implies f + I = q(f) \in P$, contradicting
the hypothesis that $P \not\in V(K)$. Thus, there exists $a \in Z(q^{-1}(P))$
such that $f(a) \neq 0 \iff f \not\in \ker{\ev_a}$. Since $q(f) = f + I \in K$,
$f \in q^{-1}K$ which means that
$q^{-1}(K) \not\subset \ker{\ev_a}$ and, by \cref{thm:correspondence-cring},
$K \not\subset q(\ker{\ev_a})$. That is,
$m_a = q(\ker{\ev_a}) \in U = \Spec(A) \setminus V(K)$.
However, since $a \in Z(q^{-1}(P))$,
$\ker{\ev_a} = I(\set{a}) \supset q^{-1}(P)$ which implies, by
\cref{thm:correspondence-cring} again, that $q(\ker{\ev_a})$ contains
$P$ and is maximal, since $\ker{\ev_a}$ is maximal. We can thus
take $M = q(\ker{\ev_a})$.
\end{proof}

\begin{lem}\label{lem:schemat-reg-fun}
Let $U$ be an affine $k$--variety with coordinate ring
$A = A(U) = R/I(U) = R/I$, where $R = k[x_1, \dots, x_n]$ and $I(U) = I$
is the ideal of $U$. Let $O_U$ be the sheaf of regular functions on $U$.
Then, we have an isomorphism of sheaves of rings:
\[
\beta_U^\sharp : O_{\Spec(A)} \to \beta_{U, *}(\alpha_{U, *}(O_U))
    : s \mapsto (a \mapsto s(\beta_U(\alpha_U(a)))(a))
\]
where $s(\beta_U(\alpha_U(a)))$ is an element $(f + I)/(g + I)$ of
$A_{\beta_U(\alpha_U(a))}$
assigned by the section $s$, for some polynomials $f, g \in R$, and
$s(\beta_U(\alpha_U(a)))(a)$ is an element of $k$ given by $f(a)/g(a)$.
In addition, for each $C \in t(U)$,
\[
\beta_{U, C}^\sharp : O_{\Spec(A), \beta_U(C)}
\to (\alpha_{U, *}(O_U))_C
\]
is a local homomorphism.
\end{lem}
\begin{proof}
Let $W$ be an open subset of $\Spec(A)$ and $s \in O_{\Spec(A)}(W)$.
Then, we have an open cover $\set{W_i}_{i \in J}$ of $W$ and
$f_i, g_i \in R$ for each $i \in J$, such that for each $P \in W_i$,
$g_i + I \not\in P$ and $s(P) = (f_i + I)/(g_i + I)$. Now, let
$a \in \alpha_{U}^{-1}(\beta_U^{-1}(W))$. Then, $\beta_U(\alpha_U(a)) \in W$
and $\cls{\set{a}} = \set{a}$ by \cref{lem:affine-var-T1}. This shows that
$\beta_U(\alpha_U(a)) = \beta_U(\cls{\set{a}}) = \beta_U(\set{a})
= q(I(\set{a}))$ where $q : R \to A$ is the quotient map.
However, $I(\set{a})$ is precisely $Z((x_1 - a_1, \dots, x_n - a_n))$,
the maximal ideal of $R$ corresponding to $a$ by \cite[Cor. I.1.4]{HartAG},
and $q(I(\set{x}))$ is the maximal ideal $m_a$ of $A$ corresponding
to $a$ by \cref{thm:correspondence-cring}. Since
$m_a = \beta_U(\alpha_U(a)) \in W$, we must have $m_a \in W_i$ for some
$i \in J$ so that $g_i + I \not\in m_a \iff g_i(a) \neq 0$. Thus,
$f_i(a)/g_i(a) \in k$. Taking $B = k$, $f = \ev_a$
$I = I$ and $J = \ker{\ev_a} = I(\set{a})$ in \cref{lem:loc-quot-ev},
we see that $f_i(a)/g_i(a) = f(a)/g(a)$ whenever
$(f_i + I)/(g_i + I) = (f + I)/(g + I)$. This, in particular, shows
that for any other $j \in J$, with $m_a \in W_j$,
$f_j(a)/g_j(a) = f_i(a)/g_i(a)$ for
$(f_j + I)/(g_j + I) = s(m_a) = (f_i + I)/(g_i + I)$.
Thus, $\beta_{U, W}^\sharp(s) : a \mapsto s(m_a)(a)$ is a well-defined function
of sets $\alpha_U^{-1}(\beta_U^{-1}(W)) \to k$. Then,
since the $W_i$ cover $W$, we have that the
$\alpha_U^{-1}(\beta_U^{-1}(W_i))$ cover $\alpha_U^{-1}(\beta_U^{-1}(W))$.
For all $i \in J, a \in \alpha_U^{-1}(\beta_U^{-1}(W_i))$,
$\beta_{U, W}^\sharp(s)(a) = f_i(a)/g_i(a)$. Hence,
$\beta_{U, W}^{\sharp}(s)$ is a regular function.

We would like to see that $\beta_{U, W}^{-1}$ is a ring homomorphism.
For $u, v \in O_{\Spec(A)}(W)$, $a \in \alpha_U^{-1}(\beta_U^{-1}(W))$,
$u(m_a) = (f_i + I)/(g_i + I), v(m_a) = (f'_i + I)/(g'_i +I)$
for some polynomials $f_i, g_i, f'_i, g'_i \in R$.
By \cref{lem:loc-quot-ev} again, we observe that:
\begin{align*}
 & \beta_{U, W}^\sharp(u + v)(a) \\
=& ((u + v)(m_a))(a) \\
=& (f_i/g_i + f'_i/g'_i)(a) \\
=& f_i(a)/g_i(a) + f'_i(a)/g'_i(a) \\
=& u(m_a)(a) + v(m_a)(a) \\
=& \beta_{U, W}^\sharp(u)(a) + \beta_{U, W}^\sharp(v)(a) \\
=& (\beta_{U, W}^\sharp + \beta_{U, W}^\sharp(v))(a)
\end{align*}
The preservation of multiplication, $0$ and $1$ can be shown by
similar point-wise identities.

We would then like to see that $\beta_U^\sharp$ is natural in $W$.
So, suppose $W' \subset W$. Then, the $W'_i = W_i \cap W'$
cover $W'$ and for $a \in \alpha_U^{-1}(\beta_U^{-1}(W'_i))$, we have:
\[
\beta_{U, W'}^\sharp(s|_{W'})(a) = s|_{W'}(m_a)(a) = s(m_a)(a)
= f_i(a)/g_i(a)
\]
which is also the value of the regular function
$\beta_{U, W}^\sharp(s)|_{\alpha_U^{-1}(\beta_U^{-1}(W'))}$.

Now, let $s(\beta_U(\alpha_U(a)))(a) = f(a)/g(a) = 0$ for
all $a \in \alpha_U^{-1}(\beta_U^{-1}(W))$. Then, $f_i/g_i$, and hence
$f_i$, is zero on all of $\alpha_U^{-1}(\beta_U^{-1}(W_i))$.
Since $\alpha_U^{-1}(\beta_U^{-1}(W_i))$ is open in $U$, it must also
be dense in $U$ by \cref{lem:open-of-irred}, since $U$ is irreducible.
Then, $f_i$ is a Zariski continuous function $U \to k$ by
\cref{lem:poly-Zar-cont}, and, hence, we must have
$f_i$ is zero on $U$, by \cref{lem:const-cont-fun-dense},
as it is zero on the dense subset
$\alpha_U^{-1}(\beta_U^{-1}(W_i)) \subset U$.
Thus, $f_i \in I$ and this means $(f_i + I)/(g_i + I)$
is the zero element of $A_P$ for each $P \in W_i$.
This shows that $s = 0$. Since $s$ was arbitrary, this shows that
$\beta^\sharp_{U, W}$ has a zero kernel and, hence, must be injective.

For surjectivity, take any regular function
$f : \alpha_U^{-1}(\beta_U^{-1}(W)) \to k$. Then, cover
$\alpha_U^{-1}(\beta_U^{-1})(W)$ with open subsets
$W_i', i \in I$ such that for each $i \in I$, there exist
$f_i, g_i \in R$ with $g_i(a) \neq 0$ for all $a \in W_i'$ and
$f|_{W_i'} = f_i/g_i$. By \cref{thm:schemat-cover}, we
have a cover $t_\Op(W_i'), i \in I$ for $\beta_U^{-1}(W)$
which, in turn, gives an open cover
$W_i := \beta_U^{-1}(t_\Op(W_i')), i \in I$
of $W$ since $\beta_U$ is a homeomorphism by
\cref{cor:schemat-affine-var-homeo}. For any $P \in W_i$, we
can show that $g_i + I \not\in P$. By \cref{lem:max-neighbour}, there exists
$a \in k^n$ with $M = q(\ker{\ev_a}) \supset P$, $M \in W_i$.
Then, $\ker{\ev_a} = q^{-1}(M) \supset q^{-1}(P)$ and
$\set{a} = Z(q^{-1}(M)) = Z(\ker{\ev_a}) \in Z(q^{-1}(W_i))
= \gamma_U(W_i) = \beta_U^{-1}(W_i)$. Then,
$a = \alpha_U^{-1}(\set{a}) \in \alpha_U^{-1}(\beta_U^{-1}(W_i'))
= W_i'$, so that $g_i(a) \neq 0$. This means, $g_i \not\in \ker{\ev_a}$
which implies that $g_i + I = q(g_i) \not\in q(\ker{\ev_a})$ and hence,
$g_i + I \not\in P$ since $P \subset q(\ker{\ev_a})$. We can then define
$s(P) = (f_i + I)/(g_i + I) \in A_P$. We need to show that $s(P)$
does not depend on the choice of $i \in I$. If $P \in W_i \cap W_j$,
then we observe that on $W_i' \cap W_j'$,
$f_i/g_i = f|_{W_i' \cap W_j'} = f_j/g_j$ so that $f_ig_j - f_jg_i = 0$
on $W_i' \cap W_j'$. Hence, $f_ig_j - f_jg_i$ is zero on $U$ by
\cref{lem:const-cont-fun-dense} and is hence in $I$. This shows
that $(f_i + I)/(g_i + I)$ and $(f_j + I)/(g_j + I)$ are equal as
elements of $A_P$.
By construction, at each $P \in W_i$,
$s(P)$ is given by the same $(f_i + I)/(g_i + I)$, and hence,
$s \in O_{\Spec(A)}(W)$. Finally, we see that
$\beta^\sharp_{U, W}(s) = f$.

To see that the localization $\beta^\sharp_{\beta_U(C)}$ is a local
homomorphism follows from the fact that it is an isomorphism, and any
isomorphism of local rings is a local homomorphism.
\end{proof}

\begin{lem}\label{lem:schemat-affine-var-local-ring}
Let $V$ be an affine $k$--variety with coordinate ring
$A = R/I$ where $R = k[x_1, \dots, x_n]$ and $I \subset R$ is a prime ideal
with $V = Z(I) \subset k^n$.
Let $C \in t(V)$.
Then, we have:
\[
(\alpha_{V_*}(O_V))_C = \colim_{C \cap W \neq \varnothing} O_V(W)
\]
where the colimit is taken over the poset of open subsets of $V$
that have non-empty intersection with $C$.
Furthermore, the set
\[
M_C := \set[{[(W, s)]}]{\exists x \in C \cap W, s(x) = 0}
\]
is the unique maximal ideal of this stalk.
\end{lem}
\begin{proof}
By \cref{thm:schematization-ord-iso} and \cref{lem:schemat-open-membership},
we have an order isomorphism between the poset of open subsets of $t(V)$
containing the point $C$ and the poset of open subsets of $V$ which have
non-empty intersection with $C$, and the first claim follows. More
explicitly, this stalk is the quotient of the set of all pairs
$(W, s)$, for an open $W \subset V$ with $C \cap W \neq \varnothing$ an
and $s \in O_V(W)$ by the equivalence relation $(W, s) \sim (W', s')$
if and only if there exists $W'' \subset W \cap W'$ such that
$s|_{W''} = s'|_{W''}$. Denote such an equivalence class as $[W, s]$
for a choice of representative $(W, s)$. By the definition of colimits,
the ring structure is given by operations
$[W, s] + [W', s'] = [W \cap W', s + s']$,
$[W, s] \cdot [W', s'] = [W \cap W', s \cdot s']$, and units
$[W, 0], [W, 1]$, where $0$ and $1$ are the constant regular functions
on $W$.

From the description of operations, it is straightforward to verify
that $M_C$ is an ideal, with the added observation that the operations
on regular functions are point-wise. If $[W, s] \not\in M_C$,
then, for all $x \in C \cap W$, $s(x) \neq 0$ and has a multiplicative
inverse $t(x) = 1/s(x)$ on $W$. This shows that the stalk quotiented
by $M_C$ is a field and hence $M_C$ is maximal. Then, suppose
$K$ is another maximal ideal of the stalk. Let $[W, s] \in K$. Then,
if $[W, s] \not\in M_C$, $[W, s]$ is a unit as we saw previously.
This means, $K$ is the stalk, which contradicts the hypothesis that
it is an ideal. Thus, $K \subset M_C$ and by maximality of $K$,
we must have $K = M_C$.
\end{proof}

\begin{thm}\label{thm:scheme-of-var}
Let $W, V$ be a $k$--varieties;
$O_W, O_V$, their sheaves of regular functions and $f : W \to V$
a morphism of varieties with $f^\sharp : O_V \to f_*(O_W)$ the map
that sends a regular function $s : V' \to k$ for some open subset
$V' \subset V$ to the regular function $s \circ f : f^{-1}(V') \to k$.
Then, in the context of \cref{cns:schematization-ringed-sp},
the following hold
\begin{enumerate}
\item $t(V, O_V)$ is a $k$--scheme.
\item $t(f, f^\sharp)$ is a morphism of $k$--schemes.
\item The functor
$t : \mathrm{RingedSp} \to \mathrm{RingedSp}$ restricted to
the category of $k$--varieties (viewed as ringed spaces with their
sheaves of regular functions) is a fully faithful functor
from $k$--varieties to $\Sch_{/k}$.
\item There is a natural transformation
$\alpha : \id_{\mathrm{RingedSp}} \to t$ such that for each $k$--variety
$V$, the map of topological spaces underlying
$\alpha_{V, O_V} : (V, O_V) \to t(V, O_V)$
is a homeomorphism onto the set of closed points of $t(V)$.
\end{enumerate}
\end{thm}
\begin{proof}
(i)
$V$ has a cover by open affine subvarieties $U_i$ by \cite[\S I.4.3]{HartAG}.
Consider the open subsets $t_\Op(U_i)$ of $t(V)$ where
$t_\Op$ is as in \cref{thm:schematization-ord-iso}.
These cover $t(V)$ by \cref{thm:schemat-cover}. Hence,
it suffices to show that each $(t_\Op(U_i), \alpha_{V, *}(O_V)|_{U_i})$
is an affine scheme.

However, we have shown in \cref{thm:schemat-open-homeo} that
for each inclusion $\iota_i : U_i \to V$, the map
$t(\iota_i) : t(U_i) \to t(V)$ is a homeomorphism onto
$t_\Op(U_i) \subset t(V)$.
For any open $W \subset t_\Op(U_i)$, we observe that
$\alpha_{V}^{-1}(W) \subset U_i$ by \cref{thm:schematization-ord-iso}
so that $\alpha_V^{-1}(W) = \iota_i^{-1}(\alpha_{V}^{-1}(W))$. Then,
by \cref{lem:schematization-nat}, we have:
\begin{align*}
\alpha_V^{-1}(W)
=& \iota_i^{-1}(\alpha_V^{-1}(W)) \\
=& (\alpha_V \circ \iota_i)^{-1}(W) \\
=& (t(\iota_i) \circ \alpha_{U_i})^{-1}(W) \\
=& \alpha_{U_i}^{-1}(t(\iota_i)^{-1}(W)
\end{align*}
Thus, we have a strict equality of sets:
\begin{align*}
 & \alpha_{V, *}(O_V)|_{t_\Op(U_i)}(W) \\
=& \alpha_{V, *}(O_V)(W) \\
=& O_V(\alpha_{V}^{-1}(W)) \\
=& O_V(\alpha_{U_i}^{-1}(t(\iota_i)^{-1}(W))) \\
=& O_V|_{U_i}(\alpha_{U_i}^{-1}(t(\iota_i)^{-1}(W))) \\
=& O_{U_i}(\alpha_{U_i}^{-1}(t(\iota_i)^{-1}(W))) \\
=& \alpha_{U_i, *}(O_{U_i})(t(\iota_i)^{-1}(W)) \\
=& t(\iota_i)_*(\alpha_{U_i, *}(O_{U_i}))(W)
\end{align*}
Therefore, we have an isomorphism of ringed spaces:
\[
(t(\iota), \id) : (t(U_i), \alpha_{U_i, *}(O_{U_i})) \to
                    (t_\Op(U_i), \alpha_{V, *}(O_V)|_{t_\Op(U_i)})
\]
We can then see that
$t(U_i, O_{U_i}) = (t(U_i), \alpha_{U_i, *}(O_{U_i}))$ is an affine scheme
by \cref{cor:schemat-affine-var-homeo}, \cref{lem:schemat-reg-fun}
and \cref{lem:set-sp-iso}.
We then need to produce a map $t(V, O_V) \to \Spec(k)$.
For this, we observe that there is a unique map of topological spaces
$f : t(V) \to |\Spec(k)| = \mathrm{pt}$, and a unique map
\[
f^\sharp_{\varnothing} : O_{\Spec(k)}(\varnothing) = \set{0}
\to \set{0} = f_*(\alpha_*(O_V))(\varnothing)
\]
We need only provide a map
\[
f^\sharp_\mathrm{pt} :
k = O_{\Spec(k)}(\mathrm{pt}) \to f_*(\alpha_*(O_V))(\mathrm{pt})
= O_V(\alpha^{-1}(f^{-1}(\mathrm{pt}))) = O_V(V)
\]
and this is simply the map giving $O_V(V)$ the structure of a $k$--algebra,
that is the map sending an element $a \in k$ to the constant regular
function on $V$ valued at $a$.

(ii)
We have to show that $t(f, f^\sharp)$ is a morphism of locally
ringed spaces over $\Spec(k)$. First, we show that it is a morphism
of ringed spaces over $\Spec(k)$. For this, we have the show that the
map
\[
k = O_{\Spec(k)}(\mathrm{pt}) \to \alpha_{W, *}(O_W)(t(W)) = O_W(W)
\]
is equal to the map:
\begin{align*}
   & k = O_{\Spec(k)}(\mathrm{pt}) \\
\to& \alpha_{V, *}(O_V)(t(V)) = O_V(V) \\
\to& t(f)_*(\alpha_{W, *}(O_W))(t(V)) = O_W(W)
\end{align*}
This follows from the observation that the constant regular function
on $V$ valued at some $a \in k$ precomposed with $f$ is the constant
regular function valued at $a$ on $W$.
Now, it suffices to show that, for each $C \in t(W)$, the map
\[
\alpha_{V, *}(f^\sharp)_{t(f)(C)} :
    \alpha_{V, *}(O_V)_{t(f)(C)} \to
    t(f)_*(\alpha_{W, *}(O_W))_{t(f)(C)} = \alpha_{W, *}(O_W)_{C}
\]
is a map of local rings.
By \cref{lem:schemat-affine-var-local-ring}, we consider
$[U, s] \in M_{t(f)(C)} \subset \alpha_{V, *}(O_V)_{t(f)(C)}$.
By definition of colimits and the map of stalks,
\[
\alpha_{V, *}(f^\sharp)_C([U, s]) = [U, \alpha_{V, *}(f^\sharp)(s)]
\]
However, the section $\alpha_{V, *}(f^\sharp)(s) \in \alpha_{V, *}(f^*(O_W))(U)$
is equal to the section
\[
s \circ f \in O_W(f^{-1}(\alpha_{V}^{-1}(U)))
\]
However, there exists $v \in t(f)(C)$ such that $s(v) \neq 0$.
If $s(f(u)) = 0$ for all $u \in f^{-1}(\alpha_V^{-1}(U))$,
then by the continuity of $s \circ f$, \cref{lem:cont-lim-pt} and
\cref{lem:affine-var-T1}, $s(u)$ must be zero since
$v \in t(f)(C) = \cls{f(C)}$,
which is a contradiction. Hence, there must be a $u \in C$ such that
$(s \circ f)(u) \neq 0$. This means,
\[
[f^{-1}(\alpha_{V}^{-1}(U)), s \circ f] \in M_C \subset \alpha_{W, *}(O_W)_C
\]
This shows that $\alpha_{V, *}(f^\sharp)_{t(f)(C)}$ is a map of local rings.

(iii)
Combine (i) and (ii) with
\cref{lem:affine-var-T1},
\cref{lem:open-cov-T1},
\cref{thm:schemat-ringed-sp-faithful},
and the observation that $T_1$ spaces are $T_0$, to obtain
that the functor is faithful. \todo[inline]{Show that it is full}

(iv)
Combine \cref{thm:schematization-fun-nat} with
\cref{thm:schematization-alpha-cont}.
\end{proof}
