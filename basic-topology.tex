
\subsection{Basic Topology}

\begin{cor}\label{cor:op-op-cls-cls}
Let $X$ be a topological spaces and $V \subset U \subset X$.
Then, if $U$ is open, then
$V$ is open in $U$ if and only if $V$ is open in
$X$. Likewise, if $U$ is closed, then
$V$ is closed in $U$ if and only if $V$
is closed in $X$.
\end{cor}
\begin{proof}
Suppose $U$ is open. Then,
$V$ is open in $U$ if and only if there exists an open
$V' \subset X$ such that $V' \cap U = V$. This implies
$V$ is open in $X$ as it is an intersection of two open
sets. If $V$ is open in $X$, then $V = V \cap U$
is an intersection of two open subsets of $U$ and hence
is open in $U$. When $U$ is closed, the proof is identical
with the word ``open'' replaced by ``closed''.
\end{proof}

\begin{lem}\label{lem:cls-int-id}
For any topological space $X$, any $U \subset X$ and
a closed $C \subset U$, $\cls{C} \cap U = C$, where the closure is taken
in $X$.
\end{lem}
\begin{proof}
Since $C \subset \cls{C}$
and $C \subset U$, we have $C \subset \cls{C} \cap U$.
Then, $C$ being closed in $U$ means there exists a closed $C' \subset X$
such that $C = C' \cap U$. In particular, $C'$ is a closed set of $X$
containing $C$ so that $\cls{C} \subset C'$ which implies that
$\cls{C} \cap U \subset C' \cap U = C$. Thus, $\cls{C} \cap U = C$.
\end{proof}

\begin{lem}\label{lem:sub-sub-topology}
Let $X$ be a topological space and $V \subset U \subset X$.
Then the subspace topology of $V$ induced from $U$
is strictly equal as a set to the subspace topology induced from $X$.
\end{lem}
\begin{proof}
Let $T_U$ and $T_X$ be the subspaces topologies on $V$
induced from $U$ and $X$ respectively.
Let $W \in T_U$ be an open set. Then, there exists an open
$W' \subset U$ such that $W = W' \cap V$.
Then, $W'$ being open in $U$ implies there exists an open
$W' \subset X$, such that $W' = W'' \cap U$. Thus,
$W = W'' \cap U \cap V = W'' \cap V$. This implies that
$W \in T_X$.

Now, suppose $W \in T_X$. Then, there exists an open $W'' \subset X$
such that $W = W'' \cap V$. Then, $W = W \cap U$ because $W \subset U$,
so that $W = W'' \cap V \cap U = (W'' \cap U) \cap V$. Here,
$W'' \cap U \subset U$ is open in $U$. Thus, $W \in T_U$.
\end{proof}

\begin{lem}\label{lem:const-cont-fun-dense}
Let $f : X \to Y$ be a contnuous function; $U \subset X$, dense; $y \in Y$,
a closed point, and $f(x) = y$ for all $x \in U$. Then, $f(x) = y$
for all $x \in X$.
\end{lem}
\begin{proof}
Then, $f^{-1}(\set{y})$ is closed subset of $X$ containing $U$, and hence
must contain $\cls{U} = X$, since $U$ is dense.
\end{proof}

\begin{lem}\label{lem:cont-lim-pt}
Let $f : X \to Y$ be a continuous function and $U \subset X$, any subset.
Suppose, there is some closed point $y \in Y$ such that $f(u) = y$ for all
$u \in U$.
Then, $f(u') = y$ for all $u' \in \cls{U}$.
\end{lem}
\begin{proof}
$f^{-1}\br{\set{y}}$ is a closed subset of $X$ containing $U$. Hence,
$\cls{U} \subset f^{-1}(\set{y})$.
\end{proof}

\begin{lem}\label{lem:cls-im}
Let $f : Y \to X$ be a continuous function and $B \subset Y$.
Then, $\cls{f(\cls{A})} = \cls{f(A)}$.
\end{lem}
\begin{proof}
$A \subset \cls{A} \implies f(A) \subset f(\cls{A})
\implies \cls{f(A)} \subset \cls{f(\cls{A})}$.
On the other hand,
$A \subset f^{-1}(f(A)) \subset f^{-1}(\cls{f(A)})$. By continuity
of $f$, $f^{-1}(\cls{f(A)})$ is closed and hence contains $\cls{A}$.
Thus, $f(\cls{A}) \subset f(f^{-1}(\cls{f(A)})) \subset \cls{f(A)}$
and hence, $\cls{f(\cls{A})} \subset \cls{f(A)}$.
\end{proof}

\begin{lem}\label{lem:open-cov-T1}
If a space $X$ has an open cover $\set{U_i}_{i \in I}$ such that
each $U_i$ is $T_1$, then $X$ is $T_1$.
\end{lem}
\begin{proof}
Let $x, y \in X$ with $x \neq y$. Then, $x \in U_i, y \in U_j$ for some
$i, j \in I$. Then, if $y \in U_i$ as well, there
are open neighbourhoods $x \in W_x \subset U_i, y \in W_y \subset U_i$,
such that $x \not\in W_y, y \not\in W_x$, and $W_x, W_y$ are open in
$X$ by \cref{cor:op-op-cls-cls}. If $x \in U_j$, a similar argument
provides open neighbourhoods $x \in W_x' \subset U_j, y \in W_y' \subset U_j$
such that $x \not\in W_y', y \not\in W_x'$.
The only remaining case is $x \not\in U_j$, $y \not\in U_i$.
In every case, we have open neighbourhoods of $x$ and $y$
not containing the other. Thus, $X$ is $T_1$.
\end{proof}
