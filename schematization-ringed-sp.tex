
\subsection{Ringed Spaces}

\begin{cor}\label{cor:pushfrwrd-schemat-nat}
In the context of \cref{cns:schematization} and \cref{lem:schematization-im},
the following diagram commutes strictly:
\[\begin{tikzcd}
\Sh(Y) \ar[r, "\alpha_{Y, *}"] \ar[d, "f_*" left] &
\Sh(t(Y)) \ar[d, "t(f)_*"] \\
\Sh(X) \ar[r, "\alpha_{X, *}" below] & \Sh(t(X))
\end{tikzcd}\]
\end{cor}
\begin{proof}
This follows from \cref{lem:schematization-nat} and the observation
that the pushforward operation $(-)_*$ commutes with composition.
\end{proof}

\begin{thm}\label{thm:alpha-pushfwd-iso}
In the context of \cref{cns:schematization} and \cref{lem:schematization-im},
the pushforward functor:
\[
\alpha_{X, *} : \Sh(X) \to \Sh(t(X))
\]
is an isomorphism (not just an equivalence) of categories.
\end{thm}
\begin{proof}
By \cref{thm:schematization-ord-iso}, $\alpha_X^{-1} : \Op(X) \to \Op(t(X))$
is an isomorphism of posets and this gives an isomorphism of categories
$(\alpha_X^{-1})^\op : \Op(X)^\op \to \Op(t(X))^\op$. This, in turn
means that the pushforward functor
\[
\alpha_{X, *} : \PSh(X) = \Fun(\Op(X)^\op, \Set)
\to \Fun(\Op(t(X))^\op, \Set) = \PSh(t(X))
\]
is an isomorphism of categories. Then, the pushforward of sheaves is given
by restricting the domain and codomain of
$\alpha_{X, *} : \PSh(X) \to \PSh(t(X))$. This shows that
$\alpha_{X, *} : \Sh(X) \to \Sh(t(X))$ is fully faithful.
We would like to show that it is strictly surjective.

Let $F : \Op(t(X))^\op \to \Set$ be a sheaf on $t(X)$.
Then, we observe that $F = F \circ t_\Op^\op \circ (\alpha_X^{-1})^\op
= \alpha_{X, *}(F \circ t_\Op^\op)$, where the equalities are strict, since
$t_\Op$ and $\alpha_X^{-1}$ are poset morphisms strictly inverse to each other.
Thus, it suffices to show that
$F \circ t_\Op^\op : \Op(X)^\op \to \Set$ is a sheaf.
Consider an open cover $\coprod_{i \in I} U_i \to U$ of $U \in \Op(X)$.
Then, $\coprod_{i \in I} t_\Op(U_i) \to t_\Op(U)$ is an open cover
of $t_\Op(U)$ by \cref{thm:schemat-cover}; for each $i, j \in I$,
$t_\Op(U_i) \cap t_\Op(U_j) = t_\Op(U_i \cap U_j)$ by
\cref{lem:schemat-ord-iso-intersection};
and since $F$ is a sheaf by assumption, the following
is an equalizer diagram:
\[\begin{tikzcd}
(F \circ t_\Op^\op)(U) \ar[r] &
\prod_{i \in I} (F \circ t_\Op^\op)(U_i)
    \ar[r, shift left] \ar[r, shift right] &
\prod_{i, j \in I} (F \circ t_\Op^\op)(U_i \cap U_j)
\end{tikzcd}\]
\end{proof}

\begin{defn}\label{defn:Set-space}
Given a category $\sC$ with products,
we call a pair $(X, E)$, where $X$ is a topological space and
$E : \Op(X)^\op \to \sC$ is a $\sC$--valued sheaf, a $\sC$--space.
A morphism of $\sC$--spaces $(Y, F) \to (X, E)$ consists of a pair
$(f, f^\sharp)$, where $f : Y \to X$ is a continuous function
and $f^\sharp : E \to f_*(F)$ is a morphism of $\sC$--valued
sheaves. Given morphisms of $\sC$--spaces
\[
(Z, G) \to[{(g, g^\sharp)}] (Y, F) \to[{(f, f^\sharp)}] (X, E)
\]
we define the composite to be the pair
$(f \circ g, f_*(g^\sharp) \circ f^\sharp)$.
One can check that $(\id_X, \id_{E})$ acts as two-sided identity
for this composition operation.
We denote
the category of $\sC$--spaces by $\Sh_\sC$. When $\sC = \Set$,
we simply write $\Sh$.
\end{defn}

\begin{lem}\label{lem:stalk-open-subset}
Let $(X, E)$ be a $\sC$--space and $U$, an open subset of $X$.
Then, for any $p \in U$, we have an isomorphism
$E_p \cong (E|_U)_p$ in $\sC$.
\end{lem}
\begin{proof}
Let $\Op(X)_p, \Op(U)_p$ denote the poset of open subsets of $X$ and $U$
containing $p$. Then, $E_p = \colim E|_{\Op(X)_p}$
and $(E|_U)_p = \colim (E|_U)_{\Op(U)_p} = \colim E|_{\Op(U)_p}$.
It suffices to show that $E_p$ is an initial cone under
$E|_{\Op(U)_p}$. We observe that for each $W \in \Op(U)_p$,
$W \in \Op(X)_p$ by \cref{cor:op-op-cls-cls} so that $E_p$
is cone under $E|_{\Op(U)_p}$ as well.
Let $\set{c_W : E(W) \to C}_{W \in \Op(U)_p}$ be another cone under
$E|_{\Op(U)_p}$.
We observe that for $S' \subset S \in \Op(X)_p$, the following diagram of
restriction maps commutes:
\[\begin{tikzcd}
E(S) \ar[r] \ar[d] & E(S \cap U) \ar[d] \\
E(S') \ar[r] & E(S' \cap U)
\end{tikzcd}\]
Therefore,
$\set{-|_{S \cap U} \circ c_{S \cap U} : E(S) \to C}_{S \in \Op(X)_p}$
is a cone under $E|_{\Op(X)_p}$.
$E_p$ being an initial cone under $E|_{\Op(X)_p}$ gives
a unique map of cones $\alpha : E_p \to C$ under $E|_{\Op(X)_p}$.
Since $\Op(U)_p \subset \Op(X)_p$, this also makes
the map $\alpha : E_p \to C$ a map of cones under $E|_{\Op(U)_p}$.
Let $\beta : E_p \to C$ be another map of cones under $E|_{\Op(U)_p}$.
Then, for any $S' \subset S \in \Op(X)_p$ again, we have that the following
diagram commutes:
\[\begin{tikzcd}
& E(S) \ar[ld] \ar[rd, "-|_{S \cap U} \circ c_{S \cap U}"] & \\
E_p \ar[rr, "\beta" below ]& & C
\end{tikzcd}\]
This makes $\beta$ a map of cones under $\Op(X)_p$ and by the uniqueness
of $\alpha$ as a map of cones under $\Op(X)_p$, we must have
$\alpha = \beta$.
\end{proof}

\begin{cns}\label{cns:schematization-ringed-sp}
Let $X$ be a topological space and $E : \Op(X)^\op \to \Set$,
a sheaf of sets on $X$. Then, define:
\[
t(X, E) := (t(X), \alpha_{X, *}(E))
\]
where $t(X), \alpha_X$ are as in \cref{cns:schematization}.
Now, consider a continuous map $f : Y \to X$, a sheaf of sets
$F$ on $Y$, and a morphism of sheaves $f^\sharp : E \to f_*(F)$.
By \cref{cor:pushfrwrd-schemat-nat},
$\alpha_{X, *}(f_*(F)) = t(f)_*(\alpha_{Y, *}(F))$. Thus, we have a map:
$\alpha_{X, *}(f^\sharp) : \alpha_{X, *}(E) \to t(f)_*(\alpha_{Y, *}(F))$.
Define:
\[
t(f, f^\sharp) := (t(f), \alpha_{X, *}(f^\sharp))
\]
In addition, we have a morphism of ringed spaces:
\[
\alpha_{(X, E)} = (\alpha_X, \id_{\alpha_{X, *}(E)}) : (X, E) \to t(X, E)
\]
\end{cns}

\begin{lem}\label{lem:schemat-ringed-sp-fun}
In the context of \cref{cns:schematization-ringed-sp}, the map
\[\begin{array}{ccccc}
t &:& \Sh &\to& \Sh \\
  &:& (X, E) &\to& (t(X), \alpha_{X, *}(E)) \\
  &:& (f, f^\sharp)
        &\to& (t(f), \alpha_{X, *}(f^\sharp)) \\
\end{array}\]
is a functor, and the maps $\alpha_{(X, F)} : (X, F) \to t(X, F)$
are natural in $(X, F)$, giving a natural transformation:
$\alpha : \id_{\mathrm{RingedSp}} \To t$.
\end{lem}
\begin{proof}
Consider a pair of composeable morphisms in $\Sh$:
\[
(Z, G) \to[{(g, g^\sharp)}] (Y, F) \to[{(f, f^\sharp)}] (X, E)
\]
We have:
\begin{align*}
 & t(f \circ g, f_*(g^\sharp) \circ f^\sharp) \\
=& (t(f \circ g), \alpha_{X, *}(f_*(g^\sharp) \circ f^\sharp)) \\
=& (t(f) \circ t(g), \alpha_{X, *}(f_*(g^\sharp) \circ f^\sharp))
    && \text{\cref{lem:schematization-fun-comp}} \\
=& (t(f) \circ t(g),
    \alpha_{X, *}(f_*(g^\sharp)) \circ \alpha_{X, *}(f^\sharp)) \\
=& (t(f) \circ t(g),
    t(f)_*(\alpha_{Y, *}(g^\sharp)) \circ \alpha_{X, *}(f^\sharp))
    && \text{\cref{cor:pushfrwrd-schemat-nat}} \\
=& (t(f), \alpha_{X, *}(f^\sharp)) \circ (t(g), \alpha_{Y, *}(g^\sharp))
\end{align*}
Then, we have:
\begin{align*}
 & t(\id_X, \id_{E}) \\
=& (t(\id_X), \alpha_{X, *}(\id_E)) \\
=& (\id_{t(X)}, \alpha_{X, *}(\id_E))
    && \text{\cref{lem:schematization-fun-id}} \\
=& (\id_{t(X)}, \id_{\alpha_{X, *}(E)})
\end{align*}

For naturality of $\alpha_{(X, E)}$ in $(X, E)$, we first observe that
that the map of underlying topological spaces $\alpha_X$ is natural in $X$
by \cref{thm:schematization-fun-nat}.
Then, we observe that the following diagram of sheaves of sets
over $X$ commutes:
\[\begin{tikzcd}[column sep=huge]
\alpha_{X, *}(E)
    \ar[r, "\alpha_{X, *}(f^\sharp)"]
    \ar[d, "\id_{\alpha_{X, *}(E)}" left] &
t(f)_*(\alpha_{Y, *}(F)) \ar[d, "\id_{t(f)_*(\alpha_{Y, *}(F))}"] \\
\alpha_{X, *}(E) \ar[r, "\alpha_{X, *}(f^\sharp)" below] &
\alpha_{X, *}(f_*(F)) = t(f)_*(\alpha_{Y, *}(F))
\end{tikzcd}\]
This shows that the following diagram of ringed spaces commutes:
\[\begin{tikzcd}
(Y, F) \ar[r, "\alpha_{Y, F}"] \ar[d, "{(f, f^\sharp)}" left] &
t(Y, F) = (t(Y), \alpha_{Y, *}(F))
    \ar[d, "{(t(f), \alpha_{X, *}(f^\sharp))}"] \\
(X, E) \ar[r, "\alpha_{X, E}" below] &
t(X, E) = (t(X), \alpha_{X, *}(E))
\end{tikzcd}\]
\end{proof}

\begin{thm}\label{thm:schemat-ringed-sp-faithful}
In the context of \cref{lem:schemat-ringed-sp-fun}, the functor $t$
restricted to the full subcategory of $\Sh$ consisting of those pairs
$(X, E)$ where $X$ is Kolmogorov ($T_0$) is faithful.
\end{thm}
\begin{proof}
Consider two maps of $\Set$--spaces (with Kolmogorov underlying spaces)
$(f, f^\sharp), (g, g^\sharp) : (Y, F) \to (X, E)$ such that
$t(f, f^\sharp) = t(g, g^\sharp)$. Then,
$t(f) = t(g)$ and $\alpha_{X, *}(f^\sharp) = \alpha_{X, *}(g^\sharp)$.
Then, \cref{thm:schemat-fun-faithful} implies $f = g$ and
\cref{thm:alpha-pushfwd-iso} implies $f^\sharp = g^\sharp$.
\end{proof}

\begin{lem}\label{lem:set-sp-iso}
If $(f, f^\sharp) : (Y, F) \to (X, E)$ is a morphism of $\Set$--spaces
such that $f$ is a homeomorphism and $f^\sharp$ is an isomorphism
of sheaves of sets over $X$, then $(f, f^\sharp)$ is an isomorphism
of $\Set$--spaces.
\end{lem}
\begin{proof}
$f$ has a continuous inverse $f^{-1} : X \to Y$. Let $W \subset Y$
be an open subset, so that $f(W) \subset X$ is also open, since $f$
is a homeomorphism. Then, $f^{-1}_*E(W) = E((f^{-1})^{-1}(W)) = E(f(W))$,
and we have an isomorphism of rings
$f^\sharp_{f(W)} : E(f(W)) \to F(f^{-1}(f(W))) = F(W)$. That is, we have the
inverse ring homomorphism
$(f^\sharp_{f(W)})^{-1} : F(W) \to E(f(W)) = f^{-1}_*E(W)$, which
we will denote as $f^{-1, \sharp}_W$. We would like to show that this
this map is natural in $U$. Let $V \subset W$, and observe that
the following diagram commutes by the naturality of $f^\sharp$:
\[\begin{tikzcd}
f^{-1}_*E(U) = E(f(W)) \ar[r, "f^\sharp_{f(U)}"] \ar[d, "-|_{f(V)}" left] &
f_*F(f(W)) = F(W) \ar[d, "-|_{f(V)}"] \\
f^{-1}_*E(V) = E(f(V)) \ar[r, "f^\sharp_{f(V)}" below] & f_*F(f(V)) = F(V)
\end{tikzcd}\]
However, since $f^{-1, \sharp}_W = (f^\sharp_W)^{-1}$ for all open
$W \subset Y$, this shows the naturality of $f^{-1, \sharp}_W$
in $W$. Hence, $(f^{-1}, f^{-1, \sharp}) : (X, E) \to (Y, F)$
is a morphism of ringed spaces. We observe that
$f_*(f^{-1, \sharp})_U = f^{-1, \sharp}_{f^{-1}(U)}$ for all
open $U \subset X$
and $f^{-1}_*(f^\sharp)_W = f^\sharp_{f(W)}$
for all open $W \subset Y$.
Then, we observe that
\[
f^{-1}_*(f^\sharp)_W \circ f^{-1, \sharp}_W
= f^{\sharp}_{f(W)} \circ (f^\sharp_{f(W)})^{-1}
= \id_{F(W)}
\]
and similarly:
\[
f_*(f^{-1, \sharp})_U \circ f^\sharp_U
= f^{-1, \sharp}_{f^{-1}(U)} \circ f^\sharp_U
= (f^{\sharp}_{U})^{-1} \circ f^\sharp_U
= \id_{E(U)}
\]
\end{proof}

\begin{lem}\label{lem:sheaf-map-localization}
Let $\alpha : E \to F$ be a map of sheaves of sets over a topological
space $X$. Then, for any $x \in X$, the map of stalks
$\alpha_x : E_x \to F_x$ is given, for any germ $[(U, s)]$ with
$x \in U, s \in E(U)$ by:
\[
[s] \mapsto {[U, \alpha_U(s)]}
\]
\end{lem}
\begin{proof}
$E_x$ is the colimit of $E$ restricted
to the subcategory of the open sets of $X$ containing $x$ so that,
we have a map $E(U) \to E_x$ for any open $U \subset X$ with $x \in U$,
and the analogous statement holds for $F_x$.
The map $\alpha_x : E_x \to F_x$ is the map obtained by the universal
property of the colimit defining $E_x$. In particular, the following
diagram commutes:
\[\begin{tikzcd}
E(U) \ar[r, "\alpha_U"] \ar[d] & F(U) \ar[d] \\
E_x \ar[r, "\alpha_x" below] & F_x
\end{tikzcd}\]
which is precisely the statement.
\end{proof}



