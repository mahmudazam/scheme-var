
\subsection{Basic Commutative Algebra Results}

\begin{lem}\label{lem:Zar-cl-field}
The Zariski closed subsets of $k$ are precisely the finite subsets.
\end{lem}
\begin{proof}
Let $S = \set{a_1, \dots, a_n} \subset k$. Then,
$S = Z((x - a_1, \dots, x - a_n))$ is a closed set.
Let $I$ be an ideal of $k[x]$ and let $f \in I$ be any polynomial.
Then, $(f) \subset I \implies Z(I) \subset Z((f))$. However,
$Z((f))$ has finitely many points as $f$ has finitely
many roots. Thus, $Z(I)$ is finite.
\end{proof}

\begin{lem}\label{lem:poly-Zar-cont}
Let $f \in k[x_1, \dots, x_n]$. Then, evaluating $f$ at points of $k^n$
defines a Zariski continuous function $k^n \to k$.
\end{lem}
\begin{proof}
By \cref{lem:Zar-cl-field}, a closed subset of $k$ is of the form
$S = \set{a_1, \dots, a_m}$. Then, $b \in f^{-1}(S) \iff f(b) - a_i = 0$ for
some $i = 1, \dots, m$. This is true if and only if
$b \in Z((f - a_1)) \cup \cdots \cup Z((f - a_m))$, which is a closed
subset of $k^n$.
\end{proof}

\begin{lem}\label{lem:ideal-preim}
Let $f : R \to S$ be a homomorphism of commutative rings.
Then, if $J \subset S$ is an ideal of $S$, $f^{-1}(J) \subset R$
is an ideal of $R$. In addition, if $J$ is prime, so is $f^{-1}(J)$
\end{lem}
\begin{proof}
We first observe that $0 \in J$ by definition of ideal, so that
$0 \in f^{-1}(J)$ by definition of ring homomorphism.
Let $a, b \in f^{-1}(J), r \in R$. We then have
$f(a - b) = f(a) - f(b) \in J$ since $f(a), f(b) \in J$ and
$J$ is an ideal. Thus, $a - b \in f^{-1}(J)$.
Next, $f(ra) = f(r)f(a) \in J$ since $f(r) \in S, f(a) \in J$
and $J$ is an ideal. Thus, $ra \in f^{-1}(J)$.
We have shown that $f^{-1}(J)$ is closed under subtraction and multiplication
by elements of $R$. Hence, $f^{-1}(J)$ is an ideal.

Suppose $J$ is prime. Then, suppose for arbitrary $p, q \in R$, we have
$pq \in f^{-1}(J)$. Then, $f(p)f(q) = f(pq) \in J$ implies that
$f(p) \in J \iff p \in f^{-1}(J)$ or $f(q) \in J \iff q \in f^{-1}(J)$.
This shows that $f^{-1}(J)$ is prime. 
\end{proof}

\begin{thm}\label{thm:correspondence-cring}
Let $R$ be a commutative ring; $I$, an ideal of $R$ and $q : R \to R/I$,
the quotient map. Let
\[
S(R, I) = \set[J]{I \subset J \subset R \text{, and $J$ is an ideal of } R}
\]
and
\[
T(R, I) = \set[J]{J \text{ is an ideal of } R/I}
\]
Then, for all $J$ in $S(R, I)$, $q(J) \in T(T, I)$;
for all $K \in T(R, I)$, $q^{-1}(K) \in S(R, I)$, and
the mappings $J \mapsto q(I)$ and $K \mapsto q^{-1}(I)$ are inverses
to each other and both preserve containment.
Finally, $J \in S(R, I)$ is prime if and only if
$q(J) \in T(R, I)$ is prime.
\end{thm}
\begin{proof}
Let $J \in S(R, I)$.
We first observe that $q(0) = 0 \in q(J)$.
Then, let $a + I, b + I \in q(J), c + I \in R/I$.
Then, there exist $a', b' \in J$ such that $q(a') = a + I, q(b') = b + I$
so that $a - a', b - b' \in I$. Then,
$(a - b) - (a' - b') = (a - a') - (b - b') \in I$ so that
$(a - b) + I = (a' - b') + I = q(a' - b')$. However,
since $J$ is an ideal, $a' - b' \in J$. This shows that
$(a + I) - (b + I) \in q(J)$. Next, $(c + I)(a +I) = ca + I$.
However, $ca - ca' = c(a - a') \in I$ since $I$ is an ideal.
Then, $ca + I = ca' + I = q(ca')$. However, $ca' \in J$ since $J$
is an ideal. This shows that $ca + I = q(ca) \in q(J)$.
Since $q(J)$ is thus closed under subtraction and multiplication
by elements of $R/I$, $q(J) \in T(R, I)$.

Now, consider $K \in T(R, I)$. Then, by \cref{lem:ideal-preim},
$q^{-1}(K)$ is and ideal of $R$. We would like to show that
$I \subset q^{-1}(K)$. Let $a \in I$, then $q(a) = a + I = 0 + I \in K$
since $K$ is an ideal. Thus, $a \in q^{-1}(\set{0}) \subset q^{-1}(K)$.
Hence, $q^{-1}(J) \in S(R, I)$.

Now, $q(q^{-1}(K)) = K$ by elementary set theory since $q$ is surjective.
Then, $J \subset q^{-1}(q(J))$, again by elementary set theory.
So, suppose $j in q^{-1}(q(J))$ so that $j + I = q(j) \in q(J)$.
This means there exists $j' \in J$ such that $q(j') = q(j)$. However,
this implies that $j' + I = j + I \iff j - j' \in I$. Since
$I \subset J$, we must have $j - j' \in J$. However, since $J$ is a
subgroup of $R$, we must have $j \in J$. Thus, $q^{-1}(q(J)) \subset J$.
Hence, we have shown that $q^{-1}(q(J)) = J$.

That $q$ and $q^{-1}$ preserve containment follows by elementary set theory.

It remains to show that $J$ is prime if and only if $q(J)$ is prime.
If $J$ is prime, suppose then that $(a + I)(b + I) = ab + I \in q(J)$.
Then, $ab + I = q(j) = j + I$ for some $j \in J$. Thus,
$ab - j \in I \subset J$ which implies that $ab \in J$ since $J$
is a subgroup of $R$. Thus, $a \in J$ or $b \in J$, since $J$ is prime
by hypothesis. Hence, $q(a) \in q(J)$ or $q(b) \in q(J)$. This shows
that $q(J)$ is prime.
If, on the other hand, $q(J)$ is prime $q^{-1}(q(J))$ is a prime
ideal of $R$ by \cref{lem:ideal-preim}. However, we have already shown
that $q^{-1}(q(J)) = J$. Hence, $J$ is prime in $R$.
\end{proof}

\begin{lem}
Let $A$ be a unital commutative ring;
$I \subset A$ an ideal with quotient map $q : A \to A/I$,
and $S \subset A$, a multiplicatively closed subset.
Then, $q(S)$ is multiplicatively closed in $A/I$.
\end{lem}
\begin{proof}
$1 \in S$ implies $q(1) = 1 + I = 1_{A/I} \in S$.
Then, let $a + I, b + I \in q(S)$. There exist
$a', b' \in S$ such that $q(a') = a' + I = a + I$
and $q(b') = b' + I = b + I$. Then, $a'b' \in S$ and
$(a + I)(b + I) = q(a')q(b') = q(a'b') \in q(S)$.
\end{proof}

\begin{lem}\label{lem:local-quot}
Let $A$ be a unital commutative ring;
$I \subset A$ an ideal with quotient map $q : A \to A/I$,
and $S \subset A$, a multiplicatively closed subset.
Then, the maps
\[
\phi : (A/I)q(S)^{-1} \to (AS^{-1})/(IS^{-1})
     : \frac{a + I}{b + I} \mapsto \frac{a}{b} + IS^{-1}
\]
and
\[
\psi : (AS^{-1})/(IS^{-1}) \to (A/I)q(S)^{-1}
     : \frac{a}{b} + IS^{-1} \mapsto \frac{a + I}{b + I}
\]
are well-defined ring homomorphisms and inverses to each other.
\end{lem}
\begin{proof}
Let $q' : AS^{-1}/IS^{-1}$ be the quotient map and
$l : A \to AS^{-1}, l' : (A/I)q(S)^{-1}$ be the localization maps.
We observe that the map
\[
A \to[l] AS^{-1} \to[q'] AS^{-1}/IS^{-1}
\]
sends the ideal $I$ to $0 + IS^{-1} = 0_{AS^{-1}/IS^{-1}}$ so that we get
a unique map $\phi'$ the following diagram commute, by the universal property
of $A/I$:
\[\begin{tikzcd}
A \ar[r, "l"] \ar[d, "q" left] & AS^{-1} \ar[d, "q'"] \\
A/I \ar[r, "\phi'" below] & AS^{-1}/IS^{-1}
\end{tikzcd}\]
This map is given by $a + I \mapsto q'(l(a)) = a/1 + IS^{-1}$.
Now, we observe that $\phi'(q(S)) = q'(l(S))$ while $l(S)$
is the set of invertible elements of $AS^{-1}$. Let $l(b) = b/1 \in l(S)$
so that $1/b \in AS^{-1}$. Then,
$q'(b/1)q'(1/b) = q'(b/1 \cdot 1/b) = q'(1) = 1$ so that $q'(l(b))$ is
invertible in $AS^{-1}/IS^{-1}$ with inverse $q'(1/b)$. This gives unique map:
\[
(A/I)q(S)^{-1} \to AS^{-1}/IS^{-1}
     : \frac{a + I}{b + I} \mapsto \frac{\phi'(a + I)}{\phi'(b + I)} + IS^{-1}
\]
However, $\phi'(a + I) = \phi'(q(a)) = q'(l(a)) = a/1 + IS^{-1}$
and similarly, $1/\phi'(b + I) = 1/b + IS^{-1}$, so that
$\frac{\phi'(a)}{\phi'(b)} + IS^{-1} = a/b + IS^{-1}$, so that this
is exactly the map $\phi$. It is a well-defined ring homomorphism
as the map was constructed by universal properties of localizations
and quotients.

Now, suppose $\phi((a + I)/(b + I)) = a/b + IS^{-1} = 0$,
then $a/b \in IS^{-1}$ which implies that $a \in I$ and $b \in S$.
However, since $a \in I$, $a + I = 0 + I$ which means
$(a + I)/(b + I) = 0$. Thus, $\phi$ is injective.
An element of $AS^{-1}/IS^{-1}$ is of the form $a/b + IS^{-1}$
which is equal to $\phi((a + I)/(b + I))$, so that $\phi$ is also
surjective. Hence, $\phi$ is a ring isomorphisms.
The argument for surjectivity also shows that the
inverse has to be $\psi$ as defined above.
\end{proof}

\begin{lem}\label{lem:loc-quot-ev}
Let $A$ be a unital commutative ring with ideals $I \subset J \subset A$
and a ring homomorphism $f : A \to B$ such that $f(A \setminus J)$
is a subset of the set of invertible elements of $B$, and $I \subset \ker{f}$.
Then, writing $q : A \to A/I$ for the
quotient map, we have a ring homomorphism:
\[
\cls{f} : (A/I)_{q(J)} \to B : \frac{a + I}{b + I} \mapsto f(a)f(b)^{-1}
\]
\end{lem}
\begin{proof}
Since $I \subset J$, $q(J)$ is an ideal of $A/I$ by
\cref{thm:correspondence-cring}.
By the universal property of quotients, we have a unique map
$f' : A/I \to B$ such that $f' \circ q = f$.
Hence,
$f'((A/I) \setminus q(J)) = f'(q(A) \setminus q(J))
\subset f'(q(A \setminus J)) = f(A \setminus J)$ since $q$ is surjective.
This shows that $f'((A/I) \setminus q(J))$ is a subset of the set of invertible
elements of $B$. Thus, writing $l : (A/I) \to (A/I)_{q(J)}$ for
the localization map, by the universal property of localization,
we get a unique map $\cls{f} : (A/I)_{q(J)} \to B$ such that
$\cls{f} \circ l = f'$ and this is given by:
\[
\cls{f}((a + I)/(b + I)) = f'(a + I)f'(b + I)^{-1} = f'(q(a))f'(q(b))^{-1}
= f(a)f(b)^{-1}
\]
\end{proof}
